\subsection{Task: DFT-embedded Local Correlation Calculations}
This tasks performs a DFT-embedded local correlation (DLPNO-coupled cluster or MP2) calculation.
This task only calls other tasks. It servers as an input helper. It calls the freeze-and-thaw/FDE task
(see Section~\ref{sec:FAT} and \ref{sec:FDE}) and local correlation task
(see Section~\ref{task:localCorrelation}).
\subsubsection{Example Input}
\begin{lstlisting}
 +task DFTEMB
   act waterA
   env waterB
   +emb
     naddXCFunc PBE
   -emb
   +lc
     method DLPNO-CCSD(T0)
   -lc
 -task
\end{lstlisting}
\subsubsection{Basic Keywords}
\begin{description}
  \item [\texttt{name}]\hfill \\
    Aliases for this task are \ttt{DFTEMBEDDEDLOCALCORRELATIONTASK}, \ttt{DFTEMBEDDING}, \ttt{DFTEMB},  and \ttt{DFTEMBLC}.
  \item [\texttt{activeSystems}]\hfill \\
    Accepts a single active system that is used for the local correlation calculation.
  \item [\texttt{environmentSystems}]\hfill \\
    Accepts a list of environment systems.
  \item [\texttt{supersystems}]\hfill \\
    Accepts a single supersystem which is used as a supersystem during the calculation.
    If none is given, a supersystem is constructed on-the-fly. The supersystem geometry and
    electronic structure will be overwritten during the calculation.
 \item [\texttt{sub-blocks}]\hfill \\
  The Embedding (\ttt{emb}) settings, local correlation (\ttt{lc}) settings,
  orbital localization task (\ttt{loc}) settings, system splitting task settings (\ttt{split})
  system addition task settings (\ttt{add}) and basis set truncation task settings (\ttt{trunc})
  are added via sub-blocks in the task settings.
  By default \ttt{splitValenceAndCore = true} is set for the localization task settings,
  \ttt{truncAlgorithm = NONE} for the basis set truncation task settings,
  \ttt{addOccupiedOrbitals = false} for the system addition task settings,
  \ttt{enforeceHFFockian = true} for the local correlation settings,
  and \ttt{embeddingMode = FERMI} for the embedding settings.
 \item [\texttt{runFaT}]\hfill \\
 If true, all subsystems are relaxed in a freeze-and-thaw procedure. By default \ttt{false}.
 \item [\texttt{fromSupersystem}]\hfill \\
 If true, the supersystem orbitals are calculated and partitioned into subsystems. By default \ttt{true}.
\end{description}

\subsubsection{Advanced Keywords}
\begin{description}
  \item [\texttt{maxCycles}]\hfill \\
  The maximum number of cycles allowed until convergence is expected by the coupled cluster iterations.
  By default \ttt{100} cycles are allowed.
  \item [\texttt{normThreshold}]\hfill \\
  The threshold at which the coupled cluster/MP2 iterations are considered converged. By default \ttt{1.0e-5}.
  \item [\texttt{writePairEnergies}]\hfill \\
  Write the pair energies to a file with name: systemName\_pairEnergies\_CCSD.dat or systemName\_pairEnergies\_MP2.dat
\end{description}
