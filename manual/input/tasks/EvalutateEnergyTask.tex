\subsection{Task: Energy Evaluation}\label{task: energy eval}
This task performs an energy evaluation for the given systems with the settings
and density currently assigned to the system. It is possible to re-evaluate the density
of systems with a different (nadd) XC functional and to use orthogonalized subsystem 
orbitals for the evaluation of $T_s^\text{nadd}$ or all energy contributions.
\subsubsection{Example Input}
\begin{lstlisting}
  +task ENERGY
    act water
    +lc
      PNOSETTINGS TIGHT
    -lc
  -task
\end{lstlisting}

\subsubsection{Basic Keywords}
\begin{description}
  \item [\texttt{name}]\hfill \\
  Aliases for this task are \ttt{EVALUATEENERGYTASK}, \ttt{ENERGYTASK}, \ttt{ENERGY}, and \ttt{E}.
  \item [\texttt{activeSystems}]\hfill \\
   Evaluates the energy for the active systems in a KS-DFT (one active system)
   or FDE (more than one active system) manner, using its density and settings.
   \item [\texttt{Supersystems}]\hfill \\
   If \ttt{evalTsOrtho} or \ttt{evalAllOrtho} is used the supersystem with the orthogonalized orbitals can be 
   stored here.
   \item [\texttt{sub-blocks}]\hfill \\
   The local correlation (\ttt{lc}) and embedding (\ttt{emb}) settings are added via sub-blocks in the task settings.
   Prominent settings in the local correlation block that are relevant for this task, and their defaults are:
    \ttt{PNOSETTINGS=TIGHT} and \ttt{method=DLPNO\_MP2}.
    Prominent settings in the embedding block that are relevant for this task, and their defaults are: \ttt{embeddingMode=NADD\_FUNC}
    and \ttt{naddXCFunc=PW91}.
    \item [\texttt{mp2Type}]\hfill \\
    MP2-type used for the evaluation of the correlation energy of double-hybrid functionals, see also \ref{task: mp2}. By default \ttt{DF}.
    \item [\texttt{maxResidual}]\hfill \\
    Maximum residual threshold for local MP2. By default \ttt{1e-5}.
    \item [\texttt{maxCycles}]\hfill \\
    Maximum number of iterations before canceling the amplitude optimization in local MP2. By default \ttt{100}.
    \item [\texttt{useDifferentXCFunc}]\hfill \\
    If \ttt{true} the XC-functional specified in \ttt{XCfunctional} (or the one in the \ttt{+customfunc} block) is used for the energy evaluation instead of the
    one specified in the system block. By default \ttt{false}.
    \item [\texttt{XCfunctional}]\hfill \\
    The XC-functional used for the evaluation of the electron density, if \ttt{useDifferentXCFunc=true}.
    By default \ttt{BP86}. This can also be customized by invoking a \ttt{+customfunc} block (see Sec.~\ref{sec:system:customfunc}). The custom functional, if specified, has higher priority than the \ttt{XCfunctional}, but it also requires \ttt{useDifferentXCFunc=true}.
	\item [\texttt{evalTsOrtho}]\hfill \\
    If enabled, the non-additive kinetic energy is evaluated from orthogonalized subsystem 
    orbitals. If \ttt{orthogonalizationScheme=NONE} the density matrix is corrected for
    the non-orthogonality of the MOs, use \ttt{evalAllOrtho} with \ttt{orthogonalizationScheme=NONE} 
    if $T_s^\text{nadd}=0$ should be calculated. By default \ttt{false}.
    \item [\texttt{evalAllOrtho}]\hfill \\
    If enabled, all energy contributions are evaluated from orthogonalized subsystem orbitals.
    By default \ttt{false}.
    \item [\texttt{orthogonalizationScheme}]\hfill \\ The orthogonalization scheme used
    for the construction of orthogonal supersystem orbitals if \ttt{evalTsOrtho} or \ttt{evalAllOrtho} are enabled.
    By default \ttt{LOEWDIN}.
\end{description}
