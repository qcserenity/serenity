\subsection{Task: Multipole Moment}
This task calculates the multipole moments of a given system, analytically or numerically.
\subsubsection{Example Input}
\begin{lstlisting}
 +task MULTIPOLEMOMENTTASK
  act water
  highestOrder 1
  origin COM
 -task
\end{lstlisting}

\subsubsection{Basic Keywords}
\begin{description}
\item [\texttt{name}]\hfill \\
  Aliases for this task are \ttt{MULTIPOLEMOMENTTASK} and \ttt{MULTI}.
\item [\texttt{activeSystems}]\hfill \\
  The systems whose multipole moments are to be calculated.
\item [\texttt{highestOrder}]\hfill \\
Calculate multipole moments up to this order (1=dipole, 2=quadrupole). 2 by default.
\item [\texttt{numerical}]\hfill \\
Multipole moments are calculated analytically using basis functions
and integrals per default (corresponding to \ttt{false}). If this is switched on,
they are calculated numerically using the density on an integration grid (corresponding to \ttt{true}).
\item [\texttt{origin}]\hfill \\
The spatial origin for the multipole calculation.
The origin of the Cartesian coordinates (\ttt{ORIGIN}, default) or the center
of mass of the given system (\ttt{COM}).
\item [\texttt{printTotal}]\hfill \\
Calculate and print the total multipole moment of all systems. The default is \ttt{false}.
Set this to \ttt{true} for \textsc{SNF}\cite{SNF2002} calculations.
\item [\texttt{printFragments}]\hfill \\
If set to \ttt{false} this suppresses the output of the multipole moment the individual subsystems.
The default is \ttt{true}.
\end{description}
