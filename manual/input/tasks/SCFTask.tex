\subsection{Task: SCF}
This task solves the Roothaan--Hall equation (Eq.~\ref{eq:roothaan-hall}) for one single system.
\begin{equation}\label{eq:roothaan-hall}
\textbf{F}\textbf{C} =  \textbf{S}\textbf{C}\varepsilon
\end{equation}
Here $\textbf{F}$ is the Fock matrix which is the matrix representation of the Fock operator $\hat{F}$:
\begin{equation}\label{eq:fockop}
\hat{F} = \hat{T}_{\text{e}} + \hat{V}_{\text{ne}} + \hat{J} + \hat{V}_{\text{xc}}~,
\end{equation}
$\textbf{S}$ is the atom orbital (AO) overlap matrix, with the elements:
\begin{equation}
S_{ij} = \langle \chi_i(r) | \chi_j(r) \rangle= \int \chi_i(r) \chi^*_j(r) ~\text{d}r,
\end{equation}
$\textbf{C}$ is the coefficient matrix containing the the linear combination of atomic orbitals (LCAO) of
each molecular orbital (MO):
\begin{equation}
\phi_i(r) = \sum_j C_{ij} \cdot \chi_j(r)~,
\end{equation}
and $\varepsilon$ is a diagonal matrix containing the orbital energies.
$\textbf{F}$ can be the (Hartree--)Fock matrix ($\hat{V}_{\text{xc}} = \hat{K}$) and also the Kohn--Sham Fock matrix ($V_{\text{xc}}[\rho(r),\nabla\rho(r),...]$), depending on the method chosen
in the \ttt{system} block of the given active system.
The iterative SCF procedure consists of three main steps:
\begin{itemize}
	\item Construction of $\textbf{F}$ from the electron density ($\textbf{P}$,$\rho(r)$)
	\item Generation of new orbitals by solving of Eq.~\ref{eq:roothaan-hall} for $\textbf{C}$
	\item The calculation of the electron density ($\textbf{D}$,$\rho(r)$) from the orbitals, using a given occupation.
\end{itemize}
Convergence of the SCF procedure is accelerated using DIIS\cite{pula1982} and damping by default.
In addition to the default settings it is also possible to switch the DIIS\cite{pula1982} for the ADIIS\cite{hu2010},
and it is possible to modify the way the damping is applied, and also its strength.
In the case of SCF calculations that do not converge easily it usually helps to simply increase the amount of damping that is
applied.
The ultimately best strategy may however vary with the given system.
For all options pertaining the convergence please see the \ttt{scf} block options of the systems, Section~\ref{sec:system:scf}.
\subsubsection{Example Input}
\begin{lstlisting}
 +task SCF
   system water
 -task
\end{lstlisting}
\subsubsection{Basic Keywords}
\begin{description}
	\item[\texttt{name}]\hfill \\
	Aliases for this task are \ttt{SCFTask} and \ttt{SCF}
	\item[\texttt{activeSystems}]\hfill \\
	Accepts a single active system that will be used in the SCF calculation.
	\item[\texttt{sub-blocks}]\hfill \\
	The local correlation (\ttt{lc}) settings can be added in a sub-block. A common setting (with its default value) in this sub-block is \ttt{pnoSettings=TIGHT}.	
	\item[\texttt{restart}]\hfill \\
	Will try to restart the SCF from loaded orbitals. By default this setting is \ttt{false}.
	\item[\texttt{mp2Type}]\hfill \\
	The MP2-type used for the evaluation of the correlation energy of double-hybrid functionals. By default \ttt{DF} is chosen, which uses the density fitting approach specified with \ttt{densfitCorr} in the system block. Other options are \ttt{AO} to evaluate the full two-electron four-center integrals and \ttt{local} to use local MP2.
	\item[\texttt{maxResidual}]\hfill \\
	Convergence threshold for the local MP2 calculation. The default value is \ttt{1.0e-5}.
	\item[\texttt{maxCycles}]\hfill \\
	Maximum number of iterations before cancelling the amplitude optimization in local MP2. The default number of cycles is \ttt{100}.
	\item[\texttt{skipSCF}]\hfill \\
	Skip the SCF procedure and perform an energy evaluation only. By default \ttt{false}.
	\item[\texttt{allowNotConverged}]\hfill \\
  If the maximum number of SCF cycles is reached Serenity will continue even with non-converged orbitals. By default \ttt{false}.
  \item[\texttt{calculateMP2Energy}]\hfill \\
  If the correlation energy of double-hybrid functionals is calculated. By default \ttt{true}.
  \item[\texttt{exca}]\hfill \\
  A vector of excitations of alpha-electrons to start a $\Delta$SCF MOM/IMOM~\cite{Gilbert2008,Barca2018} calculation. \ttt{\{0 0\}} gives a HOMO$\rightarrow$LUMO transition, \ttt{\{1 1\}} a HOMO-1$\rightarrow$LUMO+1 transition. By using \ttt{\{0\}} it is possible to do a MOM/IMOM procedure without giving a excitation. By default \ttt{\{\}}.
  \item[\texttt{excb}]\hfill \\
  The same as \texttt{exca} for beta electrons. By default \ttt{\{\}}.
  \item[\texttt{momCycles}]\hfill \\
  Number of MOM cycles before transitioning into the IMOM procedure. By default \texttt{0}.
\end{description}
